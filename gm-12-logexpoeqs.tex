% gm-12-logexpoeqs.tex

\documentclass[xcolor=dvipsnames]{beamer}

\usepackage{cancel}
\renewcommand{\CancelColor}{\color{red}}
\usepackage{graphicx}
\usepackage{wrapfig}
\usepackage{colortbl}
\usepackage{color}
\usepackage{alltt}
\renewcommand*{\thefootnote}{\fnsymbol{footnote}}
\definecolor{myblue}{rgb}{0.8,0.85,1}

\mode<presentation>
{
  \usetheme{Warsaw}
  \setbeamercovered{transparent}
}
% \usecolortheme[named=OliveGreen]{structure}
\setbeamertemplate{navigation symbols}{} 
\setbeamertemplate{blocks}[rounded][shadow=true] 

% this is for overlaying math symbols, see https://tex.stackexchange.com/questions/12895/overlay-symbol-with-another
\def\qeq{\mathrel{%
    \mathchoice{\QEQ}{\QEQ}{\scriptsize\QEQ}{\tiny\QEQ}%
}}
\def\QEQ{{%
    \setbox0\hbox{$\longrightarrow$}%
    \rlap{\hbox to \wd0{\hss/\hss}}\box0
  }}

\newcounter{expls}
\setcounter{expls}{0}
\newcommand{\beispiel}[1]{\refstepcounter{expls}\textbf{Example \arabic{expls}: #1.}}

\newcounter{exercise}
\setcounter{exercise}{0}
\newcommand{\ubung}[0]{\refstepcounter{exercise}\textbf{Exercise \arabic{exercise}: }}

\newif\ifBCITCourse
\BCITCoursetrue
% \BCITCoursefalse
\newif\ifWhichCourse
\WhichCoursetrue
\WhichCoursefalse
\ifBCITCourse
\ifWhichCourse
\newcommand{\CourseName}{Technical Mathematics for Food Technology}
\newcommand{\CourseNumber}{MATH 1441}
\newcommand{\CourseInst}{BCIT}
\else
\newcommand{\CourseName}{Technical Mathematics for Geomatics}
\newcommand{\CourseNumber}{MATH 1511}
\newcommand{\CourseInst}{BCIT}
\fi
\else
\newcommand{\CourseName}{Philosophy and Literature}
\newcommand{\CourseNumber}{PHIL 375}
\newcommand{\CourseInst}{UBC}
\fi

\title{Equations with Logarithms and Exponents}
\subtitle{{\CourseNumber}, BCIT}

\author{\CourseName}

\date{October 30, 2017}

\begin{document}

\begin{frame}
  \titlepage
\end{frame}

\begin{frame}
  \frametitle{Logarithmic Equations}
A logarithmic equation is one in which a logarithm of the variable
occurs. For example,
\begin{equation}
  \label{eq:idohchie}
  \log_{2}(x+2)=5
\end{equation}
To solve for $x$, we apply the exponential function (with the
appropriate base) to both sides of the equation. Because the
exponential function is injective (one-to-one), the resulting equation
is equivalent to the original equation,
\begin{equation}
  \label{eq:atohluek}
  x+2=2^{5}
\end{equation}
Therefore, $x=30$ and $S=\{30\}$.
\end{frame}

\begin{frame}
  \frametitle{Exercises}
{\ubung} Solve the following logarithmic equations.
  \begin{equation}
    \label{eq:vuoquohz}
\log_{2}(25-x)=3    
  \end{equation}
  \begin{equation}
    \label{eq:oaseihie}
4+3\log_{13}(2x)=16
  \end{equation}
  \begin{equation}
    \label{eq:shiexahm}
\log_{10}(x+2)+\log_{10}(x-1)=1\mbox{ (caution!)}
  \end{equation}
\end{frame}

\begin{frame}
  \frametitle{Exponential Equations}
An exponential equation is one in which the variable occurs in the
exponent. For example,
\begin{equation}
  \label{eq:ofeufowa}
  2^{x}=7
\end{equation}
Take the logarithm on both sides,
\begin{equation}
  \label{eq:cheiyohs}
  \begin{array}{rcll}
    2^{x}&=&7&\hspace{.5in}|\mbox{logarithm on both sides} \\
    \ln{}2^{x}&=&\ln{}7&\hspace{.5in}|\mbox{simplify} \\
    x\ln{}2&=&\ln{}7&\hspace{.5in}|\div{}(\ln{}2) \\
    x&=&\frac{\ln{}7}{\ln{}2}&\hspace{.5in}|\mbox{evaluate} \\
    x&\approx&2.807&
  \end{array}
\end{equation}
\end{frame}

\begin{frame}
  \frametitle{Exponential Equations Exercises}
{\ubung} Solve the following equations,
\begin{equation}
  \label{eq:faeyeije}
  3^{x+2}=7
\end{equation}
\begin{equation}
  \label{eq:ochedoxi}
  8e^{2x}=20
\end{equation}
\begin{equation}
  \label{eq:veipoeyu}
  e^{3-2x}=4
\end{equation}
\begin{equation}
  \label{eq:aquusahm}
  3x^{2}e^{x}+x^{3}e^{x}=0
\end{equation}
\end{frame}

\begin{frame}
  \frametitle{Exponential and Logarithmic Equations Exercises}
  {\ubung} Solve the following equations.
  \begin{equation}
    \label{eq:rohkiine}
    4^{1-2x}=2
  \end{equation}
  \begin{equation}
    \label{eq:zahsuini}
    8^{6+3x}=4
  \end{equation}
  \begin{equation}
    \label{eq:iaphaeya}
    3^{x^{2}+x}=\sqrt{3}
  \end{equation}
\end{frame}

\begin{frame}
  \frametitle{Exponential and Logarithmic Equations Exercises}
  {\ubung} Solve the following equations.
  \begin{equation}
    \label{eq:iebaiviu}
    4^{x-x^{2}}=\frac{1}{2}
  \end{equation}
  \begin{equation}
    \label{eq:maareiju}
    \log_{x}64=-3
  \end{equation}
  \begin{equation}
    \label{eq:sheuroov}
    \log_{\sqrt{2}}x=-6
  \end{equation}
\end{frame}

\begin{frame}
  \frametitle{Exponential and Logarithmic Equations Exercises}
  {\ubung} Solve the following equations.
  \begin{equation}
    \label{eq:dairithe}
    5^{x}=3^{x+2}
  \end{equation}
  \begin{equation}
    \label{eq:seizieng}
    5^{x+2}=7^{x-2}
  \end{equation}
  \begin{equation}
    \label{eq:ceivapuw}
    9^{2x}=27^{3x-4}
  \end{equation}
\end{frame}

\begin{frame}
  \frametitle{Exponential and Logarithmic Equations Exercises}
  {\ubung} Solve the following equations.
  \begin{equation}
    \label{eq:eefohvoh}
    25^{2x}=5^{x^{2}-12}
  \end{equation}
  \begin{equation}
    \label{eq:xiefepib}
    \log_{3}\sqrt{x-2}=2
  \end{equation}
  \begin{equation}
    \label{eq:eegaifah}
    2^{x+1}\cdot{}8^{-x}=4
  \end{equation}
\end{frame}

\begin{frame}
  \frametitle{Exponential and Logarithmic Equations Exercises}
  {\ubung} Solve the following equations.
  \begin{equation}
    \label{eq:phahmobu}
    8=4^{x^{2}}\cdot{}2^{5x}
  \end{equation}
  \begin{equation}
    \label{eq:xonguoge}
    2^{x}\cdot{}5=10^{x}
  \end{equation}
  \begin{equation}
    \label{eq:aeshaite}
    \log_{6}(x+3)+\log_{6}(x+4)=1
  \end{equation}
  \begin{equation}
    \label{eq:eequabae}
    \log(7x-12)=2\log{}x
  \end{equation}
\end{frame}

\begin{frame}
  \frametitle{Exponential and Logarithmic Equations Exercises}
  {\ubung} Solve the following equations.
  \begin{equation}
    \label{eq:niephait}
    e^{1-x}=5
  \end{equation}
  \begin{equation}
    \label{eq:eophooki}
    e^{1-2x}=4
  \end{equation}
  \begin{equation}
    \label{eq:eevaicei}
    2^{3x}=3^{2x+1}
  \end{equation}
\end{frame}

\begin{frame}
  \frametitle{Exponential and Logarithmic Equations Exercises}
  {\ubung} Solve the following equations.
  \begin{equation}
    \label{eq:ohquaiva}
    2^{x^{3}}=3^{x^{2}}
  \end{equation}
  \begin{equation}
    \label{eq:poikeeng}
    2^{\frac{2}{\log_{5}x}}=\frac{1}{16}
  \end{equation}
\begin{equation}
  \label{eq:haesiiro}
  e^{2x}-e^{x}-6=0
\end{equation}
\end{frame}

\begin{frame}
  \frametitle{End of Lesson}
Next Lesson: Conics
\end{frame}

\end{document}
