% gm-15-NormalDistribution.tex

\documentclass[xcolor=dvipsnames]{beamer}

\usepackage{tikz}
\usepackage{cancel}
\renewcommand{\CancelColor}{\color{red}}
\usepackage{graphicx}
\usepackage{wrapfig}
\usepackage{colortbl}
\usepackage{color}
\usepackage{alltt}
\renewcommand*{\thefootnote}{\fnsymbol{footnote}}
\definecolor{myblue}{rgb}{0.8,0.85,1}

\mode<presentation>
{
  \usetheme{Warsaw}
  \setbeamercovered{transparent}
}
% \usecolortheme[named=OliveGreen]{structure}
\setbeamertemplate{navigation symbols}{} 
\setbeamertemplate{blocks}[rounded][shadow=true] 

% this is for overlaying math symbols, see https://tex.stackexchange.com/questions/12895/overlay-symbol-with-another
\def\qeq{\mathrel{%
    \mathchoice{\QEQ}{\QEQ}{\scriptsize\QEQ}{\tiny\QEQ}%
}}
\def\QEQ{{%
    \setbox0\hbox{$\longrightarrow$}%
    \rlap{\hbox to \wd0{\hss/\hss}}\box0
  }}

\newcounter{expls}
\setcounter{expls}{0}
\newcommand{\beispiel}[1]{\refstepcounter{expls}\textbf{Example \arabic{expls}: #1.}}

\newcounter{exercise}
\setcounter{exercise}{0}
\newcommand{\ubung}[0]{\refstepcounter{exercise}\textbf{Exercise \arabic{exercise}: }}

\newif\ifBCITCourse
\BCITCoursetrue
% \BCITCoursefalse
\newif\ifWhichCourse
\WhichCoursetrue
\WhichCoursefalse
\ifBCITCourse
\ifWhichCourse
\newcommand{\CourseName}{Technical Mathematics for Food Technology}
\newcommand{\CourseNumber}{MATH 1441}
\newcommand{\CourseInst}{BCIT}
\else
\newcommand{\CourseName}{Technical Mathematics for Geomatics}
\newcommand{\CourseNumber}{MATH 1511}
\newcommand{\CourseInst}{BCIT}
\fi
\else
\newcommand{\CourseName}{Philosophy and Literature}
\newcommand{\CourseNumber}{PHIL 375}
\newcommand{\CourseInst}{UBC}
\fi

\title{Normal Distribution}
\subtitle{{\CourseNumber}, BCIT}

\author{\CourseName}

\date{November 22, 2017}

\begin{document}

\begin{frame}
  \titlepage
\end{frame}

\begin{frame}
  \frametitle{Probabilities}
\alert{Probabilities} are positive real numbers that add up to 1. For example,
the probabilities for heads and tails on a coin flip may be
\begin{equation}
  \label{eq:veejooni}
  P(X=H)=0.5\mbox{ and }P(X=T)=0.5
\end{equation}
If you flip two coins, what are the probabilities for getting two
heads $(X=2)$, one head(s) $(X=1)$, or none $(X=0)$?
\end{frame}

\begin{frame}
  \frametitle{Binomial Probabilities I}
It turns out that order matters, so that
\begin{equation}
  \label{eq:eijietha}
  P(X=2)=\frac{1}{3},P(X=1)=\frac{1}{3},P(X=0)=\frac{1}{3}
\end{equation}
is incorrect, while the correct distribution is
\begin{equation}
  \label{eq:eepuneeb}
  P(X=2)=\frac{1}{4},P(X=1)=\frac{1}{2},P(X=0)=\frac{1}{4}
\end{equation}
because $P(X=1)=P(\mbox{HT})+P(\mbox{TH})=1/4+1/4=1/2$.
\end{frame}

\begin{frame}
  \frametitle{Binomial Probabilities II}
Here is the formula for a \alert{binomial} setup with $n$ trials (for example,
$n=2$ coin tosses), probability of success $p$ (for example, $p=0.5$ for the
probability of heads), and $x$ number of successes,
\begin{equation}
  \label{eq:iedohdah}
  P(X=x)=\frac{n!}{(n-x)!x!}p^{x}(1-p)^{n-x}
\end{equation}
where $0!=1$ and $(n+1)!=n!(n+1)$, for example
$4!=1\cdot{}2\cdot{}3\cdot{}4$ (say ``four factorial''). $X$ is a
\alert{random variable}, the number that the random process spits out.
\end{frame}

\begin{frame}
  \frametitle{Binomial Probabilities III}
Here are the binomial probabilities for $n=6$. One way to
conceptualize these numbers is by looking at Pascal's Triangle (next
slide). 
  \begin{figure}[h]
    \includegraphics[scale=.5]{./binomial1.png}
  \end{figure}
\end{frame}

\begin{frame}
  \frametitle{Pascal's Triangle}
\begin{tikzpicture}
\foreach \n in {0,...,7} {
  \foreach \k in {0,...,\n} {
    \node at (\k-\n/2,-\n) {$\binom{\n}{\k}$};
  }
}
\end{tikzpicture}
\end{frame}

\begin{frame}
  \frametitle{Binomial Probabilities IV}
Binomial probabilities are difficult to calculate for high numbers. We
approximate the binomial distribution with the \alert{normal distribution}.
Compare the binomial distribution for $n=20$ with the normal
distribution.
  \begin{figure}[h]
    \includegraphics[scale=.5]{./binnorm.png}
  \end{figure}
\end{frame}

\begin{frame}
  \frametitle{Normal Distribution}
There is not just one normal distribution. There are infinitely many,
characterized by their \alert{mean $\mu$} and their \alert{standard deviation
$\sigma$}. The formula for the normal distribution is
\begin{equation}
  \label{eq:aitoolah}
  f(x)=\frac{1}{\sigma\sqrt{2\pi}}e^{-(x-\mu)^{2}/(2\sigma^{2})}
\end{equation}
The area under the curve tells us something about the probability of
values in the intervals in which we are interested.
  \begin{figure}[h]
    \includegraphics[scale=.4]{./qfour.png}
  \end{figure}
\end{frame}

\begin{frame}
  \frametitle{Z Scores}
To calculate the area under the curve, we carry around a piece of
paper with all the values for the \alert{standard normal distribution}
and then convert to the normal distribution with the relevant $\mu$
and $\sigma$. The value for the normal distribution is called the
\alert{$x$-score} and its associate value for the standard normal distribution
is called the \alert{$z$-score}. Travel back and forth using the following
formula,
\begin{equation}
  \label{eq:uotoogoo}
  z=\frac{x-\mu}{\sigma}
\end{equation}
\end{frame}

\begin{frame}
  \frametitle{Normal Distribution Example}
Here is an example. Men's heights are normally distributed with mean
$\mu=69.5$ inches and standard deviation $\sigma=2.4$ inches. What
percentage of the male population is taller than six feet (72 inches)?
Find the $z$-score, using the formula
\begin{equation}
  \label{eq:igutheib}
  z=\frac{x-\mu}{\sigma}=\frac{72-69.5}{2.4}\approx{}1.04
\end{equation}
Use your $z$-score table to find the corresponding \alert{$p$-value},
which is the area to the left of the $z$-score for the standard normal
distribution. In this case the $p$-value is $0.8508$. This represents
the percentage of the male population that is \emph{shorter} than 72
inches. The answer to our question is therefore, 14.92\% of men are
taller than six feet.
\end{frame}

\begin{frame}
  \frametitle{Approximating the Binomial Distribution I}
For large numbers, even high-powered computers cannot calculate the
binomial probabilities. We use the normal distribution to approximate
the binomial distribution. Here is an example: if you roll a die 600
times, what is the probability of rolling a six fewer than 80 times? It
would take very long to calculate this probability using the binomial
distribution formula! We use the normal distribution with $\mu=np$ and
$\sigma=\sqrt{np(1-p)}$ instead. We make a \alert{continuity
  correction} and ask ourselves, what is the probability for the
$x$-score to be 79.5 or less for this normal distribution?
\begin{equation}
  \label{eq:oolojuth}
  z=\frac{x-\mu}{\sigma}=\frac{79.5-100}{9.1287}\approx{}-2.25
\end{equation}
The corresponding $p$-value is 0.0122. There is only a 1.22\%
probability that you will roll fewer than 80 sixes in 600 rolls.
\end{frame}

\begin{frame}
  \frametitle{Approximating the Binomial Distribution II}
Continuity correction means that when we approximate a whole number $m$
using the continuous normal distribution, we use the interval
$[m-0.5,m+0.5]$ to represent this whole number. ``Fewer than 80,'' for
example, is translated for the approximation as ``less than 79.5.''
``Fewer than or equal to 80'' is translated as ``less than 80.5.''

Conventionally, it is only acceptable to approximate the binomial
distribution by the normal distribution if $np\geq{}5$ and
$nq\geq{}5$. Otherwise, the binomial and the normal distribution are
too far apart to provide a useful approximation.
\end{frame}

\begin{frame}
  \frametitle{Word Problems I}
The national mortality rate for a particular type of heart surgery
is 12\%, so you would expect six deaths per 50 operations. You are a
health administrator, and one of your doctors has had eleven deaths in
72 operations. Should you fire her? What is the probability that an
average surgeon (whose mortality rate is the national average) will
have twelve or more deaths in 72 operations?

\bigskip

You desperately need a person with blood type AB (incidence in
Canada: 3\%). If there are 49 people in the room, what is the
probability that at least one of them is AB?
\end{frame}

\begin{frame}
  \frametitle{Word Problems II}
  \begin{figure}[h]
    \includegraphics[scale=.7]{./triola1.png}
  \end{figure}
  \begin{figure}[h]
    \includegraphics[scale=.7]{./triola2.png}
  \end{figure}
  \begin{figure}[h]
    \includegraphics[scale=.7]{./triola3.png}
  \end{figure}
\end{frame}

\begin{frame}
  \frametitle{Word Problems III}
  \begin{figure}[h]
    \includegraphics[scale=.7]{./triola4.png}
  \end{figure}
  \begin{figure}[h]
    \includegraphics[scale=.7]{./triola5.png}
  \end{figure}
  \begin{figure}[h]
    \includegraphics[scale=.7]{./triola6.png}
  \end{figure}
\end{frame}

\begin{frame}
  \frametitle{End of Lesson}
Next Lesson: Right Spherical Trigonometry
\end{frame}

\end{document}
